\subsection{Real and Imaginary Part of a Complex Variable}

methods: {response-theory}
courses: {}

The easiest way to determine the real or imaginary part of a complex
value is to
look at it and determine the real and complex part. But if it's
not a number but an expression, this might not be straight-forwardly done.
In this case the complex conjugation of the complex number is utilized.\\

Consider a complex number $z = a +ib$ with $a$ and $b$ being the real and
imaginary part of $z$, respectively.

Real part:
\begin{equation}
 \Re(z) = \frac12 (z + z^*)
\end{equation}

\begin{align}
 \frac12 (z + z^*) &= \frac12 (a + ib + a - ib)\\
                   &= a\\
                   &= \Re(z)
\end{align}

Imaginary part:
\begin{equation}
 \Im(z) = \frac12 (z - z^*)
\end{equation}

\begin{align}
 \frac1{2i} (z - z^*) &= \frac1{2i} (a + ib - a + ib)\\
                      &= b\\
                      &= \Im(z)
\end{align}

used in: {DOI1 (equation number(s)), DOI2 (equation number(s))}
